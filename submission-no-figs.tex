\documentclass[12pt,oneside]{article}
\usepackage{lmodern}
\usepackage{amssymb,amsmath}
\usepackage{ifxetex,ifluatex}
\usepackage{fixltx2e} % provides \textsubscript
\ifnum 0\ifxetex 1\fi\ifluatex 1\fi=0 % if pdftex
  \usepackage[T1]{fontenc}
  \usepackage[utf8]{inputenc}
\else % if luatex or xelatex
  \ifxetex
    \usepackage{mathspec}
  \else
    \usepackage{fontspec}
  \fi
  \defaultfontfeatures{Ligatures=TeX,Scale=MatchLowercase}
\fi
% use upquote if available, for straight quotes in verbatim environments
\IfFileExists{upquote.sty}{\usepackage{upquote}}{}
% use microtype if available
\IfFileExists{microtype.sty}{%
\usepackage{microtype}
\UseMicrotypeSet[protrusion]{basicmath} % disable protrusion for tt fonts
}{}
\usepackage[letterpaper,margin=1in]{geometry}
\usepackage{hyperref}
\hypersetup{unicode=true,
            pdftitle={Insert title here (but don't use title case)},
            pdfkeywords={acoustics; sound; hearing},
            pdfborder={0 0 0},
            breaklinks=true}
\urlstyle{same}  % don't use monospace font for urls
\usepackage[sort&compress,super,comma]{natbib}
\bibliographystyle{bibstyle}
\usepackage{longtable,booktabs}
\usepackage{graphicx,grffile}
\makeatletter
\def\maxwidth{\ifdim\Gin@nat@width>\linewidth\linewidth\else\Gin@nat@width\fi}
\def\maxheight{\ifdim\Gin@nat@height>\textheight\textheight\else\Gin@nat@height\fi}
\makeatother
% Scale images if necessary, so that they will not overflow the page
% margins by default, and it is still possible to overwrite the defaults
% using explicit options in \includegraphics[width, height, ...]{}
\setkeys{Gin}{width=\maxwidth,height=\maxheight,keepaspectratio}
\IfFileExists{parskip.sty}{%
\usepackage{parskip}
}{% else
\setlength{\parindent}{0pt}
\setlength{\parskip}{6pt plus 2pt minus 1pt}
}
\setlength{\emergencystretch}{3em}  % prevent overfull lines
\providecommand{\tightlist}{%
  \setlength{\itemsep}{0pt}\setlength{\parskip}{0pt}}
\setcounter{secnumdepth}{5}
% Redefines (sub)paragraphs to behave more like sections
\ifx\paragraph\undefined\else
\let\oldparagraph\paragraph
\renewcommand{\paragraph}[1]{\oldparagraph{#1}\mbox{}}
\fi
\ifx\subparagraph\undefined\else
\let\oldsubparagraph\subparagraph
\renewcommand{\subparagraph}[1]{\oldsubparagraph{#1}\mbox{}}
\fi

%% %% %% %% %% %% %% %% %% %%
%%  TITLE PAGE FORMATTING  %%
%% %% %% %% %% %% %% %% %% %%
\usepackage{titling}
\pretitle{\begin{center}\large}
\posttitle{\par\end{center}\medskip}
\preauthor{\begin{center}\begin{tabular}[t]{c}}
\postauthor{\end{tabular}\par\end{center}\smallskip}
\predate{\begin{center}}
\postdate{\par\end{center}\medskip}
%% formatting for title page footnotes
\thanksmarkseries{alph}
\thanksheadextra{\thinspace}{)}
\thanksfootextra{}{)}
\setlength{\thanksmarkwidth}{1em}
\setlength{\thanksmargin}{0em}
\setlength{\footnotesep}{2em}
%% author block
\usepackage[blocks]{authblk}
\renewcommand{\Authfont}{\scshape} 
\renewcommand{\Affilfont}{\normalfont\itshape} 
\setlength{\affilsep}{0in}

%% %% %% %% %% %% %% %%
%% HEADER AND FOOTER %%
%% %% %% %% %% %% %% %%

%% %% %% %% %% %% %%
%%  LINE NUMBERS  %%
%% %% %% %% %% %% %%
\usepackage{lineno}
\renewcommand\linenumberfont{\normalfont\scriptsize}

%% %% %% %% %% %% %% %%
%%  LIST OF FIGURES  %%
%% %% %% %% %% %% %% %%
\usepackage{tocloft}
%% list of figures spacing commands
\setlength\cftbeforefigskip{1em}
\setlength{\cftafterloftitleskip}{0.5\baselineskip}
\setlength{\cftfignumwidth}{1em}
\setlength{\cftfigindent}{0pt}
\renewcommand{\cftpnumalign}{l}
\renewcommand{\cftfigpresnum}{\bfseries}
%% remove dots and page numbers from the list of figures
\renewcommand{\cftdotsep}{\cftnodots}
\cftpagenumbersoff{figure}

%% %% %% %% %% %% %% %% %%
%% FOOTNOTES > ENDNOTES %%
%% %% %% %% %% %% %% %% %%
%\usepackage{endnotes}
%\let\footnote=\endnote
%\def\enotesize{\normalsize}
%\renewcommand\enoteformat{\noindent\makebox[1em][l]{\bfseries\theenmark}\hangindent 1em}
%% remove extra space between "NOTES" and first endnote (caused by previous command)
%\apptocmd{\enoteheading}{\vspace{-\baselineskip}}{}{}
%% format endnotes name to match other headings
%\renewcommand{\notesname}{\bfseries\large\uppercase{notes}}

%% %% %% %% %% %% %% %%
%%  FORMAT HEADINGS  %%
%% %% %% %% %% %% %% %%
%% formatting for section & subsection headings and table captions
\makeatletter\renewcommand{\section}{\@startsection {section}{1}{0mm}{-\baselineskip}{0.5\baselineskip}{\bfseries\large\uppercase}}\makeatother %{name}{level}{indent}{beforeskip}{afterskip}{style}
\makeatletter\renewcommand{\subsubsection}{\@startsection {subsubsection}{3}{0mm}{-\baselineskip}{0.5\baselineskip}{\bfseries\itshape}}\makeatother %{name}{level}{indent}{beforeskip}{afterskip}{style}
\renewcommand{\thesection}{\Roman{section}.} 
\renewcommand{\thesubsection}{\Alph{subsection}.} 
\renewcommand{\thesubsubsection}{\arabic{subsubsection}.} 
%% format various other headings to match the style of section headings 
\renewcommand{\refname}{\bfseries\large\uppercase{references}}
\renewcommand{\abstractname}{\bfseries\large\uppercase{abstract}}
\renewcommand{\listfigurename}{\bfseries\large\uppercase{list of figures}}

%% %% %% %% %%
%%  TABLES  %%
%% %% %% %% %%
\usepackage{siunitx}
%\sisetup{table-number-alignment=center}
%% TABLE REFERENCE FORMATTING
\renewcommand*{\thetable}{\Roman{table}}
%% SINGLE SPACE TABLES
\usepackage{etoolbox}  % provides \AtBeginEnvironment and \apptocmd
\AtBeginEnvironment{tabular}{\singlespacing}
%% MOVE ALL TABLES TO END OF DOCUMENT
\usepackage[tablesonly, notablist, tablesfirst]{endfloat}

%% %% %% %% %% %% %% %% %%
%% OTHER CUSTOMIZATIONS %%
%% %% %% %% %% %% %% %% %%
\usepackage{setspace}  %% DOUBLE-SPACING
\usepackage{xfrac}     %% VULGAR FRACTIONS
\usepackage{fixltx2e}  %% FIX MATH IN CAPTIONS
\usepackage{textcomp}  %% for symbols that could be proper unicode, but JASA doesn't use XeLaTeX
\thickmuskip=5mu plus 3mu minus 1mu  %% ADJUST EQUATION SPACING AROUND BINARY OPERATORS LIKE =
%% SET URLS IN WHATEVER FONT SURROUNDING TEXT USES
\renewcommand{\url}{\begingroup \def\UrlLeft{}\def\UrlRight{}\urlstyle{same}\Url}
%% ADJUST EQUATION SPACING AROUND BINARY OPERATORS LIKE =
\thickmuskip=5mu plus 3mu minus 1mu
%% FONTS, ETC.
\usepackage{textcomp} % for symbols that could be done with proper unicode, but JASA doesn't use XeLaTeX
%\usepackage{fontspec}
%\setmainfont[Numbers={Lining}]{Linux Libertine O} 
%\setmonofont[ItalicFont={Source Sans Pro}]{Source Sans Pro}
%% CITATION ALIASES
%\defcitealias{ansi2004}{ANSI, 2004}  % avoid in-text citation to "American National Standards Institute"

%% %% %% %% %% %% %% %% %% %% %% %%
%% TITLE, SUBTITLE, AUTHOR, DATE %%
%% %% %% %% %% %% %% %% %% %% %% %%
\title{Insert title here (but don't use title case)}


\author{Jane Q. Author\thanks{Author to whom correspondence should be addressed. Electronic mail: jane\_q@soundu.edu}\thanksgap{1ex}}
\affil{Department of Acoustics \& Rayleigh School of Acoustical Engineering, Sound University, Sometown,
NR 54321, USA}
\author{John P. Coauthor}
\affil{Institute for Acoustics, Acoustics State University, Otherburg, RS,
98765-4321, USA}

\date{\today}

\begin{document}
\doublespacing

\maketitle\begin{center}Running title: This is the running title (6 words or fewer)\end{center}
\cleardoublepage
\linenumbers
\begin{abstract}\noindent Curabitur et finibus tellus. Mauris vel tortor non erat feugiat gravida
a ut urna. Fusce sodales odio vitae tellus luctus, a porta lectus
tempus. Nunc lacinia suscipit consectetur. Curabitur et finibus tellus.
Nunc lacinia suscipit consectetur.\newline\bigskip \copyright~2016~Acoustical Society of America\newline\bigskip\bigskip Keywords: acoustics, sound, hearing\end{abstract}\cleardoublepage

\section{Introduction}\label{introduction}

Phasellus nec ante eu ex vulputate mattis. Fusce varius maximus justo
vitae bibendum. Proin pellentesque ultrices dignissim. Class aptent
taciti sociosqu ad litora torquent per conubia nostra, per inceptos
himenaeos. Fusce ullamcorper lectus ac ipsum euismod vehicula.

Mauris vel tortor non erat feugiat gravida a ut urna. Fusce sodales odio
vitae tellus luctus, a porta lectus tempus. Nunc lacinia suscipit
consectetur. Curabitur et finibus tellus.

\section{Background}\label{background}

Suspendisse vitae feugiat augue.\citep{riesz1928} Ut sodales lobortis
sem vitae sodales.\citep{sandel1955} Interdum et malesuada fames ac ante
ipsum primis in faucibus. Etiam ultricies, elit eget venenatis aliquam,
ex ligula ultrices risus, ut varius odio purus eget ante. Curabitur
hendrerit lectus malesuada velit congue dictum.

Nam tincidunt eros eu risus elementum ultrices sit amet quis orci.
Suspendisse mattis, ante sed efficitur convallis, velit ligula lobortis
erat, quis aliquam nulla lorem sed dolor.\citep{mills1958}

\section{Methods}\label{methods}

Aenean efficitur consectetur arcu nec volutpat. Maecenas nec mattis
nunc. Phasellus sit amet elit ac dolor fermentum tempor. Aliquam ac
metus placerat, gravida tortor eget, sollicitudin ex. Nulla facilisi.

\subsection{General methods}\label{general-methods}

Curabitur hendrerit lectus malesuada velit congue dictum. Nam tincidunt
eros eu risus elementum ultrices sit amet quis orci. Suspendisse mattis,
ante sed efficitur convallis, velit ligula lobortis erat, quis aliquam
nulla lorem sed dolor. Suspendisse vitae feugiat augue.

\subsubsection{Stimuli}\label{stimuli}

Here is a cross-reference to Table~\ref{tab-simple}. In the row of
dashes in between the header row and content rows in the markdown
source, the ratio of dashes in each column determines the relative
column widths. More complicated tables (e.g., using LaTeX
\texttt{multicolumn}, or the \texttt{siunitx} package for number
formatting and alignment) can be incorporated as LaTeX tables in the
markdown source, or as separate \texttt{my-table.tex} files using
\texttt{\textbackslash{}include\{my-table\}} in the markdown source.

\begin{longtable}[]{@{}lcr@{}}
\caption{The simplest type of markdown table. There is a LaTeX
\texttt{\textbackslash{}label\{\}} tag in this caption for
cross-referencing, and a couple of \texttt{\$}-bracketed math
expressions, but otherwise the table is pure markdown.
\label{tab-simple}}\tabularnewline
\toprule
\textbf{Left aligned} & \textbf{Centered} & \textbf{Right
aligned}\tabularnewline
\midrule
\endfirsthead
\toprule
\textbf{Left aligned} & \textbf{Centered} & \textbf{Right
aligned}\tabularnewline
\midrule
\endhead
\(\beta_i \times d\thinspace^\prime\) & \texttt{foo} & some \emph{nearly} normal
text\tabularnewline
\(\alpha_j\) & bar\textsuperscript{2} & some other text\tabularnewline
\bottomrule
\end{longtable}

Ut sodales lobortis sem vitae sodales. Interdum et malesuada fames ac
ante ipsum primis in faucibus. Etiam ultricies, elit eget venenatis
aliquam, ex ligula ultrices risus, ut varius odio purus eget ante.
Curabitur hendrerit lectus malesuada velit congue dictum. Nam tincidunt
eros eu risus elementum ultrices sit amet quis orci. Suspendisse mattis,
ante sed efficitur convallis, velit ligula lobortis erat, quis aliquam
nulla lorem sed dolor. Suspendisse vitae feugiat augue.

\subsubsection{Participants}\label{participants}

Phasellus sodales, massa ac vestibulum scelerisque, mauris urna aliquet
ante, id lacinia odio enim nec sem. Nulla sit amet consequat erat.
Suspendisse ut finibus leo, ac ultricies sem. Maecenas dapibus gravida
posuere. Proin sed tincidunt turpis.

\subsubsection{Procedure}\label{procedure}

Phasellus sodales, massa ac vestibulum scelerisque, mauris urna aliquet
ante, id lacinia odio enim nec sem. Nulla sit amet consequat erat.
Suspendisse ut finibus leo, ac ultricies sem. Maecenas dapibus gravida
posuere. Proin sed tincidunt turpis.

\subsection{Experiment~1}\label{experiment-1}

Proin sed tincidunt turpis. Phasellus sodales, massa ac vestibulum
scelerisque, mauris urna aliquet ante, id lacinia odio enim nec sem. Nam
tincidunt eros eu risus elementum ultrices sit amet quis orci.
Suspendisse mattis, ante sed efficitur convallis, velit ligula lobortis
erat, quis aliquam nulla lorem sed dolor.Nulla sit amet consequat erat.
Suspendisse ut finibus leo, ac ultricies sem. Maecenas dapibus gravida
posuere.

\subsection{Experiment~2}\label{experiment-2}

Suspendisse ut finibus leo, ac ultricies sem. Maecenas dapibus gravida
posuere. Phasellus sodales, massa ac vestibulum scelerisque, mauris urna
aliquet ante, id lacinia odio enim nec sem. Nulla sit amet consequat
erat. Proin sed tincidunt turpis.

\section{Results}\label{results}

Nulla sit amet consequat erat. Phasellus sodales, massa ac vestibulum
scelerisque, mauris urna aliquet ante, id lacinia odio enim nec sem.
Suspendisse ut finibus leo, ac ultricies sem. Maecenas dapibus gravida
posuere. Nam tincidunt eros eu risus elementum ultrices sit amet quis
orci. Proin sed tincidunt turpis.

\begin{center}\bfseries [Insert Figure 1 about here]\end{center}

Suspendisse mattis, ante sed efficitur convallis, velit ligula lobortis
erat, quis aliquam nulla lorem sed dolor.

Nam tincidunt eros eu risus elementum ultrices sit amet quis orci.
Suspendisse mattis, ante sed efficitur convallis, velit ligula lobortis
erat, quis aliquam nulla lorem sed dolor.Nulla sit amet consequat erat.
Suspendisse ut finibus leo, ac ultricies sem. Maecenas dapibus gravida
posuere.

\section{Discussion}\label{discussion}

Phasellus nec ante eu ex vulputate mattis. Fusce varius maximus justo
vitae bibendum. Proin pellentesque ultrices dignissim. Class aptent
taciti sociosqu ad litora torquent per conubia nostra, per inceptos
himenaeos. Fusce ullamcorper lectus ac ipsum euismod vehicula.

Mauris vel tortor non erat feugiat gravida a ut urna. Fusce sodales odio
vitae tellus luctus, a porta lectus tempus. Nunc lacinia suscipit
consectetur. Curabitur et finibus tellus.

\cleardoublepage

%% ACKNOWLEDGMENTS MUST LATER BE MOVED BEFORE APPENDICES
\section*{acknowledgments}TODO acknowledge some folks\cleardoublepage

%% ENDNOTES
%\begingroup
%\parindent 0pt
%\parskip 1em
%\theendnotes
%\endgroup
%\cleardoublepage

\begin{thebibliography}{1}
\newcommand{\enquote}[1]{``#1''}
\providecommand{\url}[1]{\texttt{#1}}
\providecommand{\urlprefix}{URL }
\expandafter\ifx\csname urlstyle\endcsname\relax
  \providecommand{\doi}[1]{doi:\discretionary{}{}{}#1}\else
  \providecommand{\doi}{doi:\discretionary{}{}{}\begingroup
  \urlstyle{rm}\Url}\fi

\bibitem{riesz1928}
R.~R. Riesz, \enquote{Differential intensity sensitivity of the ear for pure
  tones,} Phys. Rev. \textbf{31}(5), 867--875 (1928),
  \doi{10.1103/PhysRev.31.867}.

\bibitem{sandel1955}
T.~T. Sandel, D.~C. Teas, W.~E. Feddersen, and L.~A. Jeffress,
  \enquote{Localization of sound from single and paired sources,} J. Acoust.
  Soc. Am. \textbf{27}(5), 842--852 (1955), \doi{10.1121/1.1908052}.

\bibitem{mills1958}
A.~W. Mills, \enquote{On the minimum audible angle,} J. Acoust. Soc. Am.
  \textbf{30}(4), 237--246 (1958), \doi{10.1121/1.1909553}.

\end{thebibliography}


\cleardoublepage
\processdelayedfloats
\cleardoublepage
\section*{List of figures}
\contentsline {figure}{\numberline {1}{\ignorespaces caption goes here. Mathmode test in caption: \(t_\mathrm {max}\) works.}}{6}{figure.1}

\end{document}
